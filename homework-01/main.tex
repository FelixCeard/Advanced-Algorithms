\documentclass{article}
\usepackage{import}


\title{Advanced Algorithms\\Homework 01}
\author{Felix Céard-Falkenberg\\7174020}

\usepackage{../homework} % See homework.sty %
\usepackage{amsmath}
\usepackage{amssymb} 
\usepackage[utf8]{inputenc}
\usepackage{todonotes}
\usepackage{multirow}
\usepackage{multicol}
\usepackage{booktabs}
\usepackage{graphicx}
\usepackage{float}
\usepackage{subfig}
\usepackage{bbold}
\usepackage{kantlipsum}
\usepackage[bottom]{footmisc}
\usepackage{amsthm}

\newtheorem{theorem}{Theorem}


\newcommand*\dif{\mathop{}\mathrm{d}}
\newcommand{\stepandtag}{%
  \refstepcounter{equation}%
  \tag{\theequation}%
}
\newcommand{\var}{\ensuremath{\text{Var}}}
\newcommand{\group}{\ensuremath{\text{Group}}}
\newcommand{\piml}{\ensuremath{\pi_{ML}}}
\newcommand{\variance}[1]{\ensuremath{\text{\small \textcolor{gray}{$\pm #1$}}}}
% \newcommand{\variance}[1]{\text{\tiny\textcolor{gray}{$\pm$ #1}}}


\begin{document}

\section{Question 3}

\subsection{\textit{Prove that Till's Stable Matching algorithm leads to a perfect matching and terminates, or give a counter example.}}
% We first note that we need to specify the constraint that
% We have that $|P| = |S| = \frac{|W\cup M|}{2}$, as otherwise, we only match a subset of the men and women.

\begin{theorem}
  Till's Stable Matching algorithm leads to a perfect matching and terminates.
\end{theorem}
\begin{proof}

  We can do so by using by running the Gale-Shapley algorithm with the following input:
  \begin{itemize}
    \item $X=P$
    \item $Y=S$
  \end{itemize}
  This results in a perfect matching between the individuals of $P$ and $S$, with a benefit for the individuals in $P$.

  By including individuals in $P$ from both $W$ and $M$, i.e.\ $P\cap W \neq \emptyset\land P\cap M \neq \emptyset$, we can make the algorithm more fair between the individuals from $W$ and $M$.

  The conclusion that a perfect matching between $P$ and $S$ exists, i.e., that each individual in $P$ is matched to an individual in $S$, follows from the fact that the matching returned by the Gale-Shapley algorithm is stable and perfect. 

  Furthermore, since we run the Gale-Shapley algorithm on the instance $(P, S)$, we know that it will not only terminate in $O(|P\cup S|^2) = O(|W\cup M|^2)$ time, but also that a stable matching between $P$ and $S$ is being found.

\end{proof}

\subsection{\textit{Prove that Till's algorithm leads to a stable perfect matching, or give a counter example.}}

To answer this question, we can consider two cases: the case of a stable matching between $P$ and $S$, and the case of a stable matching between $W$ and $M$.

\begin{theorem}
  Till's algorithm is $(P, S)$-stable.
\end{theorem}
\begin{proof}
  As Till's algorithm is a variant of the Gale-Shapley algorithm, we can guarantee that the algorithm returns a $(P, S)$-stable perfect matching from the properties of the Gale-Shapley algorithm.
\end{proof}

However, in the latter, we cannot guarantee that the algorithm returns a $(W, M)$-stable perfect matching. To this end, we construct an instance where Till's algorithm does not return a $(W, M)$-stable perfect matching to prove Theorem~\ref{thm:till-not-stable-w-m}.

\begin{theorem}
Till's algorithm is not $(W, M)$-stable.
\label{thm:till-not-stable-w-m}
\end{theorem}

\begin{proof}  
  Let us assume that $|W| = |M| = n$ with $n > 0$. 
  We can chose $W$ and $M$ to be arbitrary sets, as the Gale-Shapley algorithm is garantees to return a $(W, M)$-stable perfect matching when running on the stance $(W, M)$.

  Naturally, if we were to chose $P=W$ and $S=M$, we would obtain a $(W, M)$-stable perfect matching.

  However, let us consider two individuals $w\in W$ and $m\in M$ such that they are paired together in the $(W, M)$-stable perfect matching returned by the Gale-Shapley algorithm and $w'$ and $m'$ the next individuals in the preference list of $w$ and $m$, respectively. Note that we constrain $w'\in W$ and $m' \in M$.

  We can now construct our adversary instance by setting $P' = W\setminus \set{w} \cup \set{m}$ and $S' = M\setminus \set{m} \cup \set{w}$.
  We assume, for the case of the adversary instance, that the pairing between $(P', S')$ is the same as the pairing between $(W, M)$, with the exception of $w$ being paired with $w'$ and $m$ being paired with $m'$, and the original partners of $w$ and $m$ in the original pairing are now paired together in the adversary instance.

  Therefore, the returned pairing between $P'$ and $S'$ is therefore $(P', S')$-stable and perfect, however, the pair $(w, m)$ is now a blocking pair in the adversary matching.

  Therefore, due to the existence of a blocking pair in the adversary matching, and the existence of such an adversary instance, we can conclude that Till's algorithm is not $(W, M)$-stable in general. This concludes the proof.
\end{proof}

Note that Theorem~\ref{thm:till-not-stable-w-m} does not hold for all instances. For example, the instance given in the lecture notes (see Figure~\ref{fig:lecture-notes-example}) does not result in a blocking pair when switching one or more pairs between $P$ and $S$, and therefore serves as a counter example that Theorem~\ref{thm:till-not-stable-w-m} holds in all cases.
\begin{figure}[H]
  \centering
  \includegraphics[width=0.75\textwidth]{image.png}
  \caption{Example from the lecture notes.}
  \label{fig:lecture-notes-example}
\end{figure}




% \pagebreak
% \appendix
% \section{Statement on the Use of Generative AI}

Unless stated otherwise, all code, text, and math derivations have been entirely thought and written by me with no external help --- both for gen. AI and wolframalpha anc co.




\end{document}
